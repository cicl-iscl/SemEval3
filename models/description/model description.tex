\documentclass[]{article}
\usepackage{graphicx}
\usepackage{MnSymbol}
%opening
\addtolength{\oddsidemargin}{-.875in}
\addtolength{\evensidemargin}{-.875in}
\addtolength{\textwidth}{1.75in}
\title{Semeval2022 subtask3 }


\begin{document}

\maketitle


\section*{Model description}

\subsection*{General}
We used pretrained BERT (Devlin et al) networks for subtask one and two. Separate models were used for each of the languages, specifically the Alberto model (Polignano et al) was used for italian, BERT base uncased was used for english and the camembert (Martin et al) model was used for french. The BERT models are powerful and highly versatile language models that posses the benefit of being having learned good representations of the language they were trained on. This gives them a decisive edge over using models that are trained exclusively with the data provided for treining as these pretriained models will for example have encountered and learned representations for words that are not in the training set but are in the test set, whereas a model trained only on the training set will have trouble dealing with these unfamiliar words. As such they offer the opportunity for better generalization.\\
The bigger challenge of  these tasks was not to produce models that perform well on the limited training data but to produce models that generalize well and do not merely overfit on the provided data. To this end standard deep learning regularization techniques such as weight decay, dropout and model averaging were used, nonetheless the models performed much worse on the test data then on the training or validation data. Actually producing models that perform better at generalizing would likely have required data augmentation and or alternative training routines.\\
The models were not shared between subtask one and two, meaning that while the same pretrained models were used for subtask one and two the model used for subtask two was not finetuned for subtask one and vice versa. \\

\subsection*{Subtask 1}
For subtask 1 the sentences were tokenized using the tokenizers of the pretrained BERT models. The BERT model was extended with one fully connected hidden layer and an output layer. The model was trained to perform the classification task using cross entropy loss, backpropagated using the Adamw optimizer (Loshchilov et al) which combines the Adam optimizer with weight decay regularisation. Dropout was used for further regularization. Gradient clipping-by-norm was applied to solve the exploding gradient problem. Thirty percent of the data was used as a validation set, learning was terminated through early stoppage. 

\subsection*{Subtask 2}
The model for subtask 2 is similar to the model for task 1, again extending the BERT models with a hidden layer and one output layer producing a single number for the regression task. The model was trained on seventy percent of the training data using mean squared error as the loss function. Dropout was used and weight decay was applied through the AdamW optimizer. Training was terminated using early stoppage with the remaining thirty percent of the training data used for validation. For this task we trained ten models per language, each with their own training split of the data. The final prediction for the test set was the median prediction of these models. We chose to use the median and not the mean as it is less affected by outlier predictions.



\section*{quellen(noch nicht richtig formatiert)}

Jacob Devlin, Ming-Wei Chang, Kenton Lee, Kristina Toutanova; BERT: Pre-training of Deep Bidirectional Transformers for Language Understanding

Louis Martin, Benjamin Muller, Pedro Javier Ortiz Suárez, Yoann Dupont, Laurent Romary, Éric Villemonte de la Clergerie, Djamé Seddah, Benoît Sagot; CamemBERT: a Tasty French Language Model\\

Marco Polignano, Pierpaolo Basile, Marco de Gemmis, Giovanni Semeraro, Valerio Basile; ALBERTO: Italian BERT Language Understanding Model for NLP Challenging Tasks Based on Tweets; BERT: Pre-training of Deep Bidirectional Transformers for Language Understanding

Ilya Loshchilov, Frank Hutter; Decoupled Weight Decay Regularization




\end{document}



